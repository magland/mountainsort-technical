\documentclass{article}
\usepackage[utf8]{inputenc}

\title{Miscellaneous technical information about MountainSort}
\author{Jeremy Magland and Alex Barnett}
\date{February 2016}

\begin{document}

\maketitle

\section{Whitening}

The goal of whitening is to remove correlations between channels and normalize each channel to have unit variance. Let $X_{m,n}$ be the bandpassed, non-whitened signal for channel $m$ at the $n$th timepoint ($m=1,\dots,M$ and $n=1,\dots,N$). The covariance matrix is
$$C=X X',$$
which we would like to be the identity matrix. Let
$X=USV'$ be the singular value decomposition where $U$ is $M\times M$, $V$ is $N\times M$ and $S$ is the diagonal matrix of $M$ singular values.
Then the transformed (whitened) data is defined to be
$$\tilde{X}=US^{-1}U'X.$$
Note that this is equal to $UV'$ which has identity covariance matrix.

Note that $\tilde{X}$ could be replaced by $W\tilde{X}$ for any unitary transformation $W$. However, the above choice is preferred in the sense that it largely preserves the channel locality (not sure how to make this claim precise) which is important when detecting super-threshold spikes.

\section{Shell merging}

Clusters in adjacent shells are merged as follows. Let $S_j$ denote the set of events in the $j$th shell and let $C_{j,k}\subset S_j$ denote the $k$th cluster in the $j$th shell. Then we define the agreement between clusters $C_{j,k}$ and $C_{j+1,k'}$ to be
$$\eta_{j,k,j+1,k'}=\frac{\#\{A\cap B\}}{\#\{A\cup B\}}$$ where
$A=C_{j,k}\cap S_{j+1}$ and $B=C_{j+1,k'}\cap S_j$. In words, it is the fractional agreement in the overlap region of the two shells. These two clusters are then merged if 
$$\eta_{j,k,j+1,k'}>\tau$$ for some threshold $\tau\geq \frac{1}{2}$.

The merged clustering comprises clusters of the form
$$C_{j,k_1}\cup C_{j+1,k_2}\cup\dots C_{j+a-1,k_a}$$
where the adjacent clusters have been merged. This is well defined due to the following fact: If $C_{j,k}$ can be merged to at most one cluster in shell $j+1$. Indeed, suppose that it is merged to both $C_{j+1,k'}$ and $C_{j+1,k''}$. Then we have
$$\frac{\#\{A\cap B'\}}{\#A}\geq\frac{\#\{A\cap B'\}}{\#\{A\cup B''\}}>\tau\geq\frac{1}{2}$$
and
$$\frac{\#\{A\cap B''\}}{\#A}\geq\frac{\#\{A\cap B''\}}{\#\{A\cup B''\}}>\tau\geq\frac{1}{2}$$
where $A$, $B'$, and $B''$ are defined similar to as above. This implies that $B'=B''$, or equivalently $k'=k''$.


\end{document}
